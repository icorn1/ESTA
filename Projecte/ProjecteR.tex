% Options for packages loaded elsewhere
\PassOptionsToPackage{unicode}{hyperref}
\PassOptionsToPackage{hyphens}{url}
%
\documentclass[
]{article}
\usepackage{amsmath,amssymb}
\usepackage{lmodern}
\usepackage{ifxetex,ifluatex}
\ifnum 0\ifxetex 1\fi\ifluatex 1\fi=0 % if pdftex
  \usepackage[T1]{fontenc}
  \usepackage[utf8]{inputenc}
  \usepackage{textcomp} % provide euro and other symbols
\else % if luatex or xetex
  \usepackage{unicode-math}
  \defaultfontfeatures{Scale=MatchLowercase}
  \defaultfontfeatures[\rmfamily]{Ligatures=TeX,Scale=1}
\fi
% Use upquote if available, for straight quotes in verbatim environments
\IfFileExists{upquote.sty}{\usepackage{upquote}}{}
\IfFileExists{microtype.sty}{% use microtype if available
  \usepackage[]{microtype}
  \UseMicrotypeSet[protrusion]{basicmath} % disable protrusion for tt fonts
}{}
\makeatletter
\@ifundefined{KOMAClassName}{% if non-KOMA class
  \IfFileExists{parskip.sty}{%
    \usepackage{parskip}
  }{% else
    \setlength{\parindent}{0pt}
    \setlength{\parskip}{6pt plus 2pt minus 1pt}}
}{% if KOMA class
  \KOMAoptions{parskip=half}}
\makeatother
\usepackage{xcolor}
\IfFileExists{xurl.sty}{\usepackage{xurl}}{} % add URL line breaks if available
\IfFileExists{bookmark.sty}{\usepackage{bookmark}}{\usepackage{hyperref}}
\hypersetup{
  pdftitle={Projecte d'R per ESTA},
  pdfauthor={Ixent Cornella},
  hidelinks,
  pdfcreator={LaTeX via pandoc}}
\urlstyle{same} % disable monospaced font for URLs
\usepackage[margin=1in]{geometry}
\usepackage{color}
\usepackage{fancyvrb}
\newcommand{\VerbBar}{|}
\newcommand{\VERB}{\Verb[commandchars=\\\{\}]}
\DefineVerbatimEnvironment{Highlighting}{Verbatim}{commandchars=\\\{\}}
% Add ',fontsize=\small' for more characters per line
\usepackage{framed}
\definecolor{shadecolor}{RGB}{248,248,248}
\newenvironment{Shaded}{\begin{snugshade}}{\end{snugshade}}
\newcommand{\AlertTok}[1]{\textcolor[rgb]{0.94,0.16,0.16}{#1}}
\newcommand{\AnnotationTok}[1]{\textcolor[rgb]{0.56,0.35,0.01}{\textbf{\textit{#1}}}}
\newcommand{\AttributeTok}[1]{\textcolor[rgb]{0.77,0.63,0.00}{#1}}
\newcommand{\BaseNTok}[1]{\textcolor[rgb]{0.00,0.00,0.81}{#1}}
\newcommand{\BuiltInTok}[1]{#1}
\newcommand{\CharTok}[1]{\textcolor[rgb]{0.31,0.60,0.02}{#1}}
\newcommand{\CommentTok}[1]{\textcolor[rgb]{0.56,0.35,0.01}{\textit{#1}}}
\newcommand{\CommentVarTok}[1]{\textcolor[rgb]{0.56,0.35,0.01}{\textbf{\textit{#1}}}}
\newcommand{\ConstantTok}[1]{\textcolor[rgb]{0.00,0.00,0.00}{#1}}
\newcommand{\ControlFlowTok}[1]{\textcolor[rgb]{0.13,0.29,0.53}{\textbf{#1}}}
\newcommand{\DataTypeTok}[1]{\textcolor[rgb]{0.13,0.29,0.53}{#1}}
\newcommand{\DecValTok}[1]{\textcolor[rgb]{0.00,0.00,0.81}{#1}}
\newcommand{\DocumentationTok}[1]{\textcolor[rgb]{0.56,0.35,0.01}{\textbf{\textit{#1}}}}
\newcommand{\ErrorTok}[1]{\textcolor[rgb]{0.64,0.00,0.00}{\textbf{#1}}}
\newcommand{\ExtensionTok}[1]{#1}
\newcommand{\FloatTok}[1]{\textcolor[rgb]{0.00,0.00,0.81}{#1}}
\newcommand{\FunctionTok}[1]{\textcolor[rgb]{0.00,0.00,0.00}{#1}}
\newcommand{\ImportTok}[1]{#1}
\newcommand{\InformationTok}[1]{\textcolor[rgb]{0.56,0.35,0.01}{\textbf{\textit{#1}}}}
\newcommand{\KeywordTok}[1]{\textcolor[rgb]{0.13,0.29,0.53}{\textbf{#1}}}
\newcommand{\NormalTok}[1]{#1}
\newcommand{\OperatorTok}[1]{\textcolor[rgb]{0.81,0.36,0.00}{\textbf{#1}}}
\newcommand{\OtherTok}[1]{\textcolor[rgb]{0.56,0.35,0.01}{#1}}
\newcommand{\PreprocessorTok}[1]{\textcolor[rgb]{0.56,0.35,0.01}{\textit{#1}}}
\newcommand{\RegionMarkerTok}[1]{#1}
\newcommand{\SpecialCharTok}[1]{\textcolor[rgb]{0.00,0.00,0.00}{#1}}
\newcommand{\SpecialStringTok}[1]{\textcolor[rgb]{0.31,0.60,0.02}{#1}}
\newcommand{\StringTok}[1]{\textcolor[rgb]{0.31,0.60,0.02}{#1}}
\newcommand{\VariableTok}[1]{\textcolor[rgb]{0.00,0.00,0.00}{#1}}
\newcommand{\VerbatimStringTok}[1]{\textcolor[rgb]{0.31,0.60,0.02}{#1}}
\newcommand{\WarningTok}[1]{\textcolor[rgb]{0.56,0.35,0.01}{\textbf{\textit{#1}}}}
\usepackage{graphicx}
\makeatletter
\def\maxwidth{\ifdim\Gin@nat@width>\linewidth\linewidth\else\Gin@nat@width\fi}
\def\maxheight{\ifdim\Gin@nat@height>\textheight\textheight\else\Gin@nat@height\fi}
\makeatother
% Scale images if necessary, so that they will not overflow the page
% margins by default, and it is still possible to overwrite the defaults
% using explicit options in \includegraphics[width, height, ...]{}
\setkeys{Gin}{width=\maxwidth,height=\maxheight,keepaspectratio}
% Set default figure placement to htbp
\makeatletter
\def\fps@figure{htbp}
\makeatother
\setlength{\emergencystretch}{3em} % prevent overfull lines
\providecommand{\tightlist}{%
  \setlength{\itemsep}{0pt}\setlength{\parskip}{0pt}}
\setcounter{secnumdepth}{-\maxdimen} % remove section numbering
\ifluatex
  \usepackage{selnolig}  % disable illegal ligatures
\fi

\title{Projecte d'R per ESTA}
\author{Ixent Cornella}
\date{24/12/2021}

\begin{document}
\maketitle

\hypertarget{projecte-destadistica}{%
\section{PROJECTE D'ESTADISTICA}\label{projecte-destadistica}}

\#\#\#Introducció

Aquest projecte implementat en R, té com a objectiu demostrar
coneixement adquirit durant el curs sobre estadística descrpitiva i
infèrencia estadística, en diferents problemes, i en general, resoldre
problemes creats a partir de dades inventades.

Les dades que s'utilizaran pels problemes provenen de un fitxer excel,
que mostra comandes sobre menjar. Es faran analisis sobre les dades, per
comprovar les seves propietats estadistiques.

\hypertarget{obtenim-les-dades-per-la-practica}{%
\subsection{Obtenim les dades per la
practica:}\label{obtenim-les-dades-per-la-practica}}

Primer ens calen alguns paquets per poder dexifrar el fitxer excel en R

\begin{Shaded}
\begin{Highlighting}[]
  \FunctionTok{library}\NormalTok{(XLConnect)}
\end{Highlighting}
\end{Shaded}

\begin{verbatim}
## Warning: package 'XLConnect' was built under R version 4.1.2
\end{verbatim}

\begin{verbatim}
## XLConnect 1.0.5 by Mirai Solutions GmbH [aut],
##   Martin Studer [cre],
##   The Apache Software Foundation [ctb, cph] (Apache POI),
##   Graph Builder [ctb, cph] (Curvesapi Java library),
##   Brett Woolridge [ctb, cph] (SparseBitSet Java library)
\end{verbatim}

\begin{verbatim}
## https://mirai-solutions.ch
## https://github.com/miraisolutions/xlconnect
\end{verbatim}

Aquest paquet ens servira per poder tractar el fitxer excel en format de
data frame.

La comanda readWorksheetFromFile llegeix una fitxa de excel i la
transforma en un dataframe de R.

\begin{Shaded}
\begin{Highlighting}[]
\NormalTok{  data }\OtherTok{=} \FunctionTok{readWorksheetFromFile}\NormalTok{(}\StringTok{"sampledatafoodsales.xlsx"}\NormalTok{, }\AttributeTok{sheet =} \DecValTok{2}\NormalTok{)}
\end{Highlighting}
\end{Shaded}

\#\#Contingut de les dades Abans de començar, revisem una mica les dades
del data frame \emph{data} que hem creat:

\begin{Shaded}
\begin{Highlighting}[]
\FunctionTok{names}\NormalTok{(data) }\CommentTok{\# retorna el nom de les variables}
\end{Highlighting}
\end{Shaded}

\begin{verbatim}
## [1] "OrderDate"  "Region"     "City"       "Category"   "Product"   
## [6] "Quantity"   "UnitPrice"  "TotalPrice"
\end{verbatim}

Podem observar com tenim varies columnes. La primera indica la data en
la que es va fer la comanda, la segona la regió, la ciutat, la categoria
a la que pertany el producte, el producte ensí, la quantitat que s'ha
demanat d'aquest, el preu per unitat i el preu total. És bastant segur
que no utilitzarem columnes com la de la data en la que es va fer la
comanda, ja que ens interesen altres tipus de variables, com poden ser
les numeriques o de tipus \emph{string}. Anem a veure el tipus de
variables:

\begin{Shaded}
\begin{Highlighting}[]
\FunctionTok{sapply}\NormalTok{(data,class) }\CommentTok{\# retorna el nom de les variables}
\end{Highlighting}
\end{Shaded}

\begin{verbatim}
## $OrderDate
## [1] "POSIXct" "POSIXt" 
## 
## $Region
## [1] "character"
## 
## $City
## [1] "character"
## 
## $Category
## [1] "character"
## 
## $Product
## [1] "character"
## 
## $Quantity
## [1] "numeric"
## 
## $UnitPrice
## [1] "numeric"
## 
## $TotalPrice
## [1] "numeric"
\end{verbatim}

\hypertarget{primera-tasca-observar-si-hi-ha-una-distribucio-normal-entre-els-preus-de-cada-producte-test-de-normalitat.}{%
\section{\texorpdfstring{\textbf{Primera tasca}: Observar si hi ha una
distribucio normal entre els preus de cada producte (Test de
normalitat).}{Primera tasca: Observar si hi ha una distribucio normal entre els preus de cada producte (Test de normalitat).}}\label{primera-tasca-observar-si-hi-ha-una-distribucio-normal-entre-els-preus-de-cada-producte-test-de-normalitat.}}

\begin{Shaded}
\begin{Highlighting}[]
\NormalTok{preu}\OtherTok{\textless{}{-}}\NormalTok{data}\SpecialCharTok{$}\NormalTok{UnitPrice}
 \FunctionTok{hist}\NormalTok{(preu, }\AttributeTok{breaks =} \FunctionTok{seq}\NormalTok{(}\FloatTok{1.3}\NormalTok{, }\FloatTok{3.5}\NormalTok{, }\AttributeTok{by=} \FloatTok{0.1}\NormalTok{))}
\end{Highlighting}
\end{Shaded}

\includegraphics{ProjecteR_files/figure-latex/p1-1.pdf}

\begin{Shaded}
\begin{Highlighting}[]
\CommentTok{\# Es pot veure clarament al histograma que no hi ha cap tipus d\textquotesingle{}indici per pensar que els preus estan distribuits de forma normal.}
\CommentTok{\# tot i aixo, ens assegurarem amb les funcions d\textquotesingle{}R}
\FunctionTok{qqnorm}\NormalTok{(preu)}
\FunctionTok{qqline}\NormalTok{(preu)}
\end{Highlighting}
\end{Shaded}

\includegraphics{ProjecteR_files/figure-latex/p1-2.pdf}

\begin{Shaded}
\begin{Highlighting}[]
\CommentTok{\#Ara es pot veure clarament com els valors no segueixen cap tipus de relacio normal, si aixi fos, veuriem els puntets negres seguint la linia al grafic.}
\FunctionTok{shapiro.test}\NormalTok{(preu)}
\end{Highlighting}
\end{Shaded}

\begin{verbatim}
## 
##  Shapiro-Wilk normality test
## 
## data:  preu
## W = 0.76733, p-value < 2.2e-16
\end{verbatim}

La H0 del test és que els x estan distribuïts segons una normal. Com que
p \textless\textless{} 0.02, hem de rebutjar H0 i concloure que les
dades no corresponen a una normal.

\hypertarget{segona-tasca-veure-com-a-partir-de-dades-de-variables-independents-es-pot-obtenir-dades-distribuides-segons-una-normal.}{%
\section{Segona Tasca: Veure com a partir de dades de variables
independents, es pot obtenir dades distribuides segons una
normal.}\label{segona-tasca-veure-com-a-partir-de-dades-de-variables-independents-es-pot-obtenir-dades-distribuides-segons-una-normal.}}

El TCL diu que si sumem les dades corresponents a moltes variables
independents, obtenim unes dades que estan aproximadament distribuïdes
segons una normal. Podem comprovar-ho amb alguna variable del fitxer
excel obtingut.

\begin{Shaded}
\begin{Highlighting}[]
\NormalTok{p }\OtherTok{=} \FunctionTok{mean}\NormalTok{(preu}\DecValTok{{-}2}\NormalTok{) }\CommentTok{\#per tenir un valor entre 0 i 1}
\NormalTok{n }\OtherTok{=} \FunctionTok{as.integer}\NormalTok{(}\FunctionTok{mean}\NormalTok{(data}\SpecialCharTok{$}\NormalTok{TotalPrice))}
\NormalTok{mu }\OtherTok{=}\NormalTok{ n}\SpecialCharTok{*}\NormalTok{p}
\NormalTok{sigma }\OtherTok{=} \FunctionTok{sqrt}\NormalTok{(n}\SpecialCharTok{*}\NormalTok{p}\SpecialCharTok{*}\NormalTok{(}\DecValTok{1}\SpecialCharTok{{-}}\NormalTok{p))}
\NormalTok{h}\OtherTok{=}\DecValTok{1}\SpecialCharTok{/}\NormalTok{(sigma}\SpecialCharTok{*}\FunctionTok{sqrt}\NormalTok{(}\DecValTok{2}\SpecialCharTok{*}\NormalTok{pi)) }\CommentTok{\#h es el valor maxim de la funció de densitat de la normal}
\end{Highlighting}
\end{Shaded}

Dibuixem la funcio de densitat de la binomial.

\begin{Shaded}
\begin{Highlighting}[]
\NormalTok{k}\OtherTok{=}\DecValTok{0}\SpecialCharTok{:}\NormalTok{n}
\NormalTok{f}\OtherTok{\textless{}{-}}\FunctionTok{dbinom}\NormalTok{(k,n,p) }\CommentTok{\#funcio de densitat (d)}
\FunctionTok{plot}\NormalTok{(k,f,}\AttributeTok{type=}\StringTok{\textquotesingle{}h\textquotesingle{}}\NormalTok{, }\AttributeTok{col=}\StringTok{"blue"}\NormalTok{,}\AttributeTok{ylim=}\DecValTok{0}\SpecialCharTok{:}\FloatTok{1.1}\SpecialCharTok{*}\NormalTok{h)}
\CommentTok{\#text}
\NormalTok{tt}\OtherTok{=}\FunctionTok{sprintf}\NormalTok{(}\StringTok{"Funcio de densitat}\SpecialCharTok{\textbackslash{}n}\StringTok{ d\textquotesingle{}una Bi(\%d, \%1.1f), amb valors del fitxer \textquotesingle{}foodsales.xlsx\textquotesingle{}"}\NormalTok{,n,p)}
\FunctionTok{text}\NormalTok{(}\AttributeTok{x=}\FloatTok{0.75}\SpecialCharTok{*}\NormalTok{n, }\AttributeTok{y=}\FloatTok{0.65}\SpecialCharTok{*}\NormalTok{h, }\AttributeTok{labels=}\NormalTok{tt, }\AttributeTok{cex=}\DecValTok{1}\NormalTok{, }\AttributeTok{col=}\StringTok{"blue"}\NormalTok{)}
\CommentTok{\#fi text}
\NormalTok{x}\OtherTok{=}\FunctionTok{seq}\NormalTok{(mu}\DecValTok{{-}4}\SpecialCharTok{*}\NormalTok{sigma, mu}\SpecialCharTok{+}\DecValTok{4}\SpecialCharTok{*}\NormalTok{sigma, }\AttributeTok{by=}\FloatTok{0.01}\SpecialCharTok{*}\NormalTok{sigma)}
\NormalTok{f}\OtherTok{\textless{}{-}}\FunctionTok{dnorm}\NormalTok{(x,mu,sigma)}
\FunctionTok{lines}\NormalTok{(x,f,}\AttributeTok{type=}\StringTok{"l"}\NormalTok{, }\AttributeTok{col=}\StringTok{"red"}\NormalTok{)}
\CommentTok{\#text}
\NormalTok{tt}\OtherTok{=}\FunctionTok{sprintf}\NormalTok{(}\StringTok{"Funcio de densitat}\SpecialCharTok{\textbackslash{}n}\StringTok{ d\textquotesingle{}una N(\%1.1f, \%1.1f)"}\NormalTok{,mu,sigma)}
\FunctionTok{text}\NormalTok{(}\AttributeTok{x=}\FloatTok{0.65}\SpecialCharTok{*}\NormalTok{n, }\AttributeTok{y=}\FloatTok{0.85}\SpecialCharTok{*}\NormalTok{h, }\AttributeTok{labels=}\NormalTok{tt, }\AttributeTok{cex=}\DecValTok{1}\NormalTok{, }\AttributeTok{col=}\StringTok{"red"}\NormalTok{)}
\end{Highlighting}
\end{Shaded}

\includegraphics{ProjecteR_files/figure-latex/unnamed-chunk-3-1.pdf}

Tenim np = 27, n(1 − p) = 108, Valors molt per sobre de 5 (si estan per
sobre donen una bona aproximacio).

Finalment, podem comprovar aquests valors del preu total de cada
comanda:

\begin{Shaded}
\begin{Highlighting}[]
\FunctionTok{summary}\NormalTok{(data}\SpecialCharTok{$}\NormalTok{TotalPrice)}
\end{Highlighting}
\end{Shaded}

\begin{verbatim}
##    Min. 1st Qu.  Median    Mean 3rd Qu.    Max. 
##   33.60   72.57  102.75  136.58  159.30  817.92
\end{verbatim}

I mirar-ho graficament per veure que molts valors d'aquest preu total
estan agrupats al voltant de la mediana, amb molts valors atipics.

\begin{Shaded}
\begin{Highlighting}[]
\FunctionTok{boxplot}\NormalTok{(data}\SpecialCharTok{$}\NormalTok{TotalPrice)}
\end{Highlighting}
\end{Shaded}

\includegraphics{ProjecteR_files/figure-latex/unnamed-chunk-5-1.pdf}

\hypertarget{tercera-tasca-esbrinar-per-cada-tipus-lindex-de-quantitat-daquest.}{%
\section{Tercera Tasca: Esbrinar, per cada tipus, l'index de quantitat
d'aquest.}\label{tercera-tasca-esbrinar-per-cada-tipus-lindex-de-quantitat-daquest.}}

L'index de quantitat ens l'inventem. Aquest estarà representat de la
següent forma: el total de productes i quantes unitats d'aquests
productes es demanen per comanda. Quants més es demanen (en mitjana),
més popular és el producte (és a dir, es demanen més quantitats
d'aquests producte per comanda. Normalment els productes petits com ara
galetes acostumen a ser més demanats en comanda. Veurem si és cert)

Primer, ens cal esbrinar la freqüencia de cada producte. Es pot fer amb
les següents comandes:

\begin{Shaded}
\begin{Highlighting}[]
\NormalTok{freq }\OtherTok{=} \FunctionTok{table}\NormalTok{(data}\SpecialCharTok{$}\NormalTok{Product)}
\NormalTok{freq\_rel }\OtherTok{=} \FunctionTok{prop.table}\NormalTok{(freq)}
\FunctionTok{round}\NormalTok{(freq\_rel}\SpecialCharTok{*}\DecValTok{100}\NormalTok{,}\AttributeTok{digits=}\DecValTok{2}\NormalTok{)}
\end{Highlighting}
\end{Shaded}

\begin{verbatim}
## 
##      Arrowroot         Banana           Bran         Carrot Chocolate Chip 
##          12.70           1.23          11.07          26.23          13.52 
## Oatmeal Raisin   Potato Chips       Pretzels    Whole Wheat 
##          12.70           9.02           2.87          10.66
\end{verbatim}

\begin{Shaded}
\begin{Highlighting}[]
\NormalTok{labs }\OtherTok{\textless{}{-}} \FunctionTok{names}\NormalTok{(}\FunctionTok{table}\NormalTok{(data}\SpecialCharTok{$}\NormalTok{Product))}
\FunctionTok{par}\NormalTok{(}\AttributeTok{mar=}\FunctionTok{c}\NormalTok{(}\DecValTok{7}\NormalTok{,}\DecValTok{4}\NormalTok{,}\DecValTok{2}\NormalTok{,}\DecValTok{2}\NormalTok{)) }\CommentTok{\#Ajustem marges}
\FunctionTok{barplot}\NormalTok{(}\AttributeTok{height=}\NormalTok{freq\_rel, }\AttributeTok{col=}\StringTok{"\#80B1FC"}\NormalTok{, }\AttributeTok{names.arg=}\NormalTok{labs, }\AttributeTok{las=}\DecValTok{2}\NormalTok{)}
\end{Highlighting}
\end{Shaded}

\includegraphics{ProjecteR_files/figure-latex/unnamed-chunk-6-1.pdf} Com
es pot veure, es demanen moltes carrotes i pocs platàns. Això no
influeix l'index de quantitat, ja que volem esbrinar quants d'aquests
productes es demanen per comanda.

Ara ens cal trobar el total de cada producte que s'ha demanat en totes
les comandes

\begin{Shaded}
\begin{Highlighting}[]
\NormalTok{X }\OtherTok{=} \FunctionTok{c}\NormalTok{(}\DecValTok{0}\NormalTok{,}\DecValTok{0}\NormalTok{,}\DecValTok{0}\NormalTok{,}\DecValTok{0}\NormalTok{,}\DecValTok{0}\NormalTok{,}\DecValTok{0}\NormalTok{,}\DecValTok{0}\NormalTok{,}\DecValTok{0}\NormalTok{,}\DecValTok{0}\NormalTok{)}

\CommentTok{\#recorrem tot el dataframe i anem sumant les quantitats de les comandes en funció del producte}

\ControlFlowTok{for}\NormalTok{(i }\ControlFlowTok{in} \DecValTok{1}\SpecialCharTok{:}\FunctionTok{nrow}\NormalTok{(data))\{}
  \ControlFlowTok{if}\NormalTok{(data[i,}\StringTok{"Product"}\NormalTok{] }\SpecialCharTok{==} \StringTok{"Arrowroot"}\NormalTok{)\{}
\NormalTok{    X[}\DecValTok{1}\NormalTok{] }\OtherTok{=}\NormalTok{ X[}\DecValTok{1}\NormalTok{] }\SpecialCharTok{+}\NormalTok{ data[i,}\StringTok{"Quantity"}\NormalTok{]}
\NormalTok{  \}}
  \ControlFlowTok{else} \ControlFlowTok{if}\NormalTok{(data[i,}\StringTok{"Product"}\NormalTok{] }\SpecialCharTok{==} \StringTok{"Banana"}\NormalTok{)\{}
\NormalTok{    X[}\DecValTok{2}\NormalTok{] }\OtherTok{=}\NormalTok{ X[}\DecValTok{2}\NormalTok{] }\SpecialCharTok{+}\NormalTok{ data[i,}\StringTok{"Quantity"}\NormalTok{]}
\NormalTok{  \}}
  \ControlFlowTok{else} \ControlFlowTok{if}\NormalTok{(data[i,}\StringTok{"Product"}\NormalTok{] }\SpecialCharTok{==} \StringTok{"Bran"}\NormalTok{)\{}
\NormalTok{    X[}\DecValTok{3}\NormalTok{] }\OtherTok{=}\NormalTok{ X[}\DecValTok{3}\NormalTok{] }\SpecialCharTok{+}\NormalTok{ data[i,}\StringTok{"Quantity"}\NormalTok{]}
\NormalTok{  \}}
  \ControlFlowTok{else} \ControlFlowTok{if}\NormalTok{(data[i,}\StringTok{"Product"}\NormalTok{] }\SpecialCharTok{==} \StringTok{"Carrot"}\NormalTok{)\{}
\NormalTok{    X[}\DecValTok{4}\NormalTok{] }\OtherTok{=}\NormalTok{ X[}\DecValTok{4}\NormalTok{] }\SpecialCharTok{+}\NormalTok{ data[i,}\StringTok{"Quantity"}\NormalTok{]}
\NormalTok{  \}}
  \ControlFlowTok{else} \ControlFlowTok{if}\NormalTok{(data[i,}\StringTok{"Product"}\NormalTok{] }\SpecialCharTok{==} \StringTok{"Chocolate Chip"}\NormalTok{)\{}
\NormalTok{    X[}\DecValTok{5}\NormalTok{] }\OtherTok{=}\NormalTok{ X[}\DecValTok{5}\NormalTok{] }\SpecialCharTok{+}\NormalTok{ data[i,}\StringTok{"Quantity"}\NormalTok{]}
\NormalTok{  \}}
  \ControlFlowTok{else} \ControlFlowTok{if}\NormalTok{(data[i,}\StringTok{"Product"}\NormalTok{] }\SpecialCharTok{==} \StringTok{"Oatmeal Raisin"}\NormalTok{)\{}
\NormalTok{    X[}\DecValTok{6}\NormalTok{] }\OtherTok{=}\NormalTok{ X[}\DecValTok{6}\NormalTok{] }\SpecialCharTok{+}\NormalTok{ data[i,}\StringTok{"Quantity"}\NormalTok{]}
\NormalTok{  \}}
  \ControlFlowTok{else} \ControlFlowTok{if}\NormalTok{(data[i,}\StringTok{"Product"}\NormalTok{] }\SpecialCharTok{==} \StringTok{"Potato Chips"}\NormalTok{)\{}
\NormalTok{    X[}\DecValTok{7}\NormalTok{] }\OtherTok{=}\NormalTok{ X[}\DecValTok{7}\NormalTok{] }\SpecialCharTok{+}\NormalTok{ data[i,}\StringTok{"Quantity"}\NormalTok{]}
\NormalTok{  \}}
  \ControlFlowTok{else} \ControlFlowTok{if}\NormalTok{(data[i,}\StringTok{"Product"}\NormalTok{] }\SpecialCharTok{==} \StringTok{"Pretzels"}\NormalTok{)\{}
\NormalTok{    X[}\DecValTok{8}\NormalTok{] }\OtherTok{=}\NormalTok{ X[}\DecValTok{8}\NormalTok{] }\SpecialCharTok{+}\NormalTok{ data[i,}\StringTok{"Quantity"}\NormalTok{]}
\NormalTok{  \}}
  \ControlFlowTok{else} \ControlFlowTok{if}\NormalTok{(data[i,}\StringTok{"Product"}\NormalTok{] }\SpecialCharTok{==} \StringTok{"Whole Wheat"}\NormalTok{)\{}
\NormalTok{    X[}\DecValTok{9}\NormalTok{] }\OtherTok{=}\NormalTok{ X[}\DecValTok{9}\NormalTok{] }\SpecialCharTok{+}\NormalTok{ data[i,}\StringTok{"Quantity"}\NormalTok{]}
\NormalTok{  \}}
\NormalTok{\}}
\NormalTok{index }\OtherTok{=}\NormalTok{ X }\SpecialCharTok{*}\NormalTok{ freq\_rel}
\CommentTok{\#Frequencia relativa}
\NormalTok{freq\_rel}
\end{Highlighting}
\end{Shaded}

\begin{verbatim}
## 
##      Arrowroot         Banana           Bran         Carrot Chocolate Chip 
##     0.12704918     0.01229508     0.11065574     0.26229508     0.13524590 
## Oatmeal Raisin   Potato Chips       Pretzels    Whole Wheat 
##     0.12704918     0.09016393     0.02868852     0.10655738
\end{verbatim}

\begin{Shaded}
\begin{Highlighting}[]
\CommentTok{\#Index de productes}
\NormalTok{index}
\end{Highlighting}
\end{Shaded}

\begin{verbatim}
## 
##      Arrowroot         Banana           Bran         Carrot Chocolate Chip 
##    310.6352459      0.9713115    174.2827869   1098.2295082    330.6762295 
## Oatmeal Raisin   Potato Chips       Pretzels    Whole Wheat 
##    327.0245902     89.6229508      5.3360656    101.9754098
\end{verbatim}

\emph{index} representa, per cada producte, el total de unitats
comandades d'aquest, multiplicat per la frequencia en la que s'ordenen
les comandes en funció del producte. Si \emph{index} és un valor molt
alt, vol dir que s'han fet comandes on s'han ordenat molts d'aquests
productes, com pot ser per \emph{carrota}, el seu index és de gairebé
1100, cosa que correspon amb la freqüencia absoluta de les carrotes, que
és la més alta. Com es pot observar, quan més baixa és la freqüencia
absoluta d'un producte, més baix acostuma a ser l'índex. Podem veure
excepcions, com ara \emph{Whole Wheat} i \emph{Bran}, que tenen una
frequencia absoluta similar, però el index de \emph{Bran} és més alt que
el de \emph{Whole Wheat}, cosa que ens indica que per s'ha demanat més
quantitat de \emph{Bran} que de \emph{Whole Wheat} en totes les
comandes.

\#Conclusió

En aquest projecte s'han usat els coneixements adquirits al labortari
d'Estadística, per intentar demostrar-los. S'han fet tècniques tant
d'inferència estadística (test de normalitat, TCL) com d'estadistica
descrpitiva. També cal destacar que per obtenir les dades s'ha utlitzat
una funció no apresa a classe, amb una llibreria que permet transformar
fitxers excel en format de data frame de R.

\end{document}
